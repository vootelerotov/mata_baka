\documentclass[a4paper]{article}

\usepackage[a4paper]{geometry}
\usepackage{amssymb,amsmath}
\usepackage[utf8]{inputenc}
\usepackage{enumerate}
\usepackage{color}
\usepackage{graphicx}
\usepackage[estonian]{babel}
\usepackage[usenames,dvipsnames]{xcolor}
\usepackage{float}
\usepackage{cite}
\usepackage{etoolbox}
\usepackage[hidelinks]{hyperref}

\makeatletter
\patchcmd{\HyField@FlagsRadioButton}{\HyField@SetFlag{Ff}{Radio}}{}{}{}
\makeatother
\def\DefaultOptionsofRadio{print}



\definecolor{background_example}{HTML}{E0DCDE}



\setlength\parindent{0pt}
\newenvironment{tightcenter}{%
  \setlength\topsep{0pt}
  \setlength\parskip{0pt}
  \begin{center}
}{%
  \end{center}
}

\author{Vootele Rõtov}
\title{Bakatöö}
\begin{document}

\maketitle

\pagebreak

\tableofcontents

\pagebreak

\section*{Spikker}

{\color{cyan} Helesinine- komentaarid}

{\color{blue} Tumesinine - asjad, mille õigsust peaks kontrollima}


{\color{cyan}\section*{Miks ma käesolevat asja teen?}

Panen kirja mõned põhjused, miks ma tegelen selle asjaga:
\begin{enumerate}[I]
\item Sellest on kellegile kasu - loodan realselt, et saan hakkama mingi toreda asjaga, millest keegi(näiteks Marguse psühholoogidest sõbrad) kasu saab.
\item Saab kätte selle paberi, mis teeb minust parema inimese. 
\item Midagi uut, olen juba päris pikalt tarkust ühes formaadis kuula õppejõudu, tööta läbi tema poolt valitud materjalid, esita see õppejõule, äkki selline formaat, kus tuleb ise otsida ja ise mõelda meeldib.
\item Väljakutse, kuigi viimaste aastatega on enesemotivatsioon kõvasti paranenud, ei ole see veel seal, kus ta olla võiks. Loodetavasti "treenin" seda aspekti.
\end{enumerate}}

\section{Töö \"ulesehitus}
Esiteks, toome sisse lugejale vajaliku taustinfo, seejärel kirjeldame probleemip\"ustitust. Sellele järgneb töö eesmärgi matemaatiline p\"ustitus.

\section{Taustinfo}

Käesolev uurimus tegeleb k\"simustikega, milles soovitakse hinnanguid mingile arvule väidetele viie pallisel Likerti skaalal. Näiteks: \footnote{Näited terviklikest k\"usimustikest lisades, \hyperref[likert1]{joonis \ref*{likert1}} ja \hyperref[likert2]{joonis \ref*{likert2}}}

\vspace{10pt}

\begin{figure}[H]


\colorbox{background_example}{\parbox{\textwidth}{

\vspace{1mm}

Käesoleva bakalaureusetöö \"ulesehitus on loogiline.

\vspace{5pt}

\begin{Form}
\def\DefaultWidthofChoiceMenu{12pt}%



\ChoiceMenu[bordercolor = gray,disabled = true,name=optionE,radio,radiosymbol=\ding{108}]{\mbox{}}\null Ei nõustu 
\ChoiceMenu[bordercolor = gray,disabled = true,name=optionD,radio,radiosymbol=\ding{108}]{\mbox{}}\null Ei nõustu osaliselt
\ChoiceMenu[bordercolor = gray,disabled = true,name=optionC,radio,radiosymbol=\ding{108}]{\mbox{}}\null Nii ja naa
\ChoiceMenu[bordercolor = gray,disabled = true, name=optionB,radio,radiosymbol=\ding{108}]{\mbox{}}\null Nõustun osaliselt
\ChoiceMenu[bordercolor = gray,disabled = true,name=optionA,radio,radiosymbol=\ding{108}]{\mbox{}}\null Nõustun


\end{Form}}}
\caption{Näide väitest, millile palutakse hinnangut Likerti skaalal}
\label{likert_question}
\end{figure}








\section{Mida me teen?}


\begin{figure}[H]


\colorbox{background_example}
	{\parbox
		{\textwidth}
			{
			\setlength{\unitlength}{1mm}
			\begin{picture}(148,15)
				\put(0,12){Käesoleva bakalaureusetöö \"ulesehitus on loogiline.}
				\put(0,7){\line(1,0){14}}
				\put(8,1){Ei nõustu}
				\put(16,7){\circle{4}}
				\put(16,7){\circle*{2}}
				\put(18,7){\line(1,0){25}}
				\put(32,1){Ei nõustu osaliselt}
				\put(45,7){\circle{4}}
				\put(45,7){\circle*{2}}
				\put(47,7){\line(1,0){25}}
				\put(67,1){Nii ja Naa}
				\put(74,7){\circle{4}}
				\put(74,7){\circle*{2}}
				\put(76,7){\line(1,0){25}}
				\put(92,1){Nõustun osaliselt}
				\put(103,7){\circle{4}}
				\put(103,7){\circle*{2}}
				\put(105,7){\line(1,0){25}}
				\put(126,1){Nõustun}
				\put(132,7){\circle{4}}
				\put(132,7){\circle*{2}}
				\put(134,7){\vector(1,0){14}}
			\end{picture}
		}
		
	}
\caption{Näide, kuidas hinnangud skaalal vaikimisi paigutuvad}
\label{quiz}
\end{figure}

\begin{figure}


	\colorbox{background_example}
	{\parbox
		{\textwidth}
			{
			\setlength{\unitlength}{1mm}
			\begin{picture}(148,15)
				\put(0,12){Käesoleva bakalaureusetöö \"ulesehitus on loogiline.}
				\put(0,7){\line(1,0){8}}
				\put(2,1){Ei nõustu}
				\put(10,7){\circle{4}}
				\put(10,7){\circle*{2}}
				\put(12,7){\line(1,0){18}}
				\put(19,1){Ei nõustu osaliselt}
				\put(32,7){\circle{4}}
				\put(32,7){\circle*{2}}
				\put(34,7){\line(1,0){20}}
				\put(49,1){Nii ja Naa}
				\put(56,7){\circle{4}}
				\put(56,7){\circle*{2}}
				\put(58,7){\line(1,0){33}}
				\put(82,1){Nõustun osaliselt}
				\put(93,7){\circle{4}}
				\put(93,7){\circle*{2}}
				\put(95,7){\line(1,0){30}}
				\put(121,1){Nõustun}
				\put(127,7){\circle{4}}
				\put(127,7){\circle*{2}}
				\put(129,7){\vector(1,0){19}}
			\end{picture}
		}
	}

\caption{Näide hinnangute paiknemisest skaalal,kus on arvestatud hindajate  suhtumist autorisse}
\label{quiz}

\end{figure}



Meil on valikvastutsega küsimustik milles on $n$ küsimust, kõik küsimusid 5 palli skaalal. Vaja on mudelit, mis aitaks anal\"u\"usida k\"usimuste skaalat ehk seda, kas hindamise skaalal asuvad k\"simused sama kaugel. Selline mudel on kasutatav k\"simustike täiendamiseks. 

Olemas on mitmeid erinevaid lahendusi. Margus pakkus välja järgmise mõtte : 

\textit{Kolmas mõte oleks aga lineariseerida muutuja valides kordajad "optimaalselt" nii, et kokku oleks (lineaarsete s.o. pearsoni) korellatsioonide summa maksimaalne.}

Minu \"ulesanne: analüüsida, kas see lahendus idee on mõistlik ja kui on mõistlik, siis välja mõelda, kuidas see lahendus täpselt võiks välja näha. 

\section*{Altervatiivid}

\begin{enumerate}
\item Kõige triviaalsem viis: kõikide k\"usimuste vastused kujutama hulgale \{1,2,3,4,5\}, leiame mudeli, mille võime kirjeldada valitud k\"usimust on suurim. Tegemist on ilmselt vaikimisi variandiga ehk loodud mudelit peab võrdlema 
\item Selle asemel, et kujutada hulgale {1,2,3,4,5}, leiame sobivad vasted nii, et mudeli kirjeldav jõud oleks suurim. Oht selles, et meie mudel kirjeldab väga hästi olemasolevat valimit, \"utlemata suurt midagi \"uldkogumi kohta. {\color{blue} Huvitav oleks, kas lihsalt nii midagi teha ei annaks ? Overfittimise vastu saaks, aga vaja oleks valimit, mis oleks piisavalt suur, et seda kaheks jagada(midagi, mille pealt mudelit ehitada ja midagi, mille pealt seda validifitseerida}. 
\item {\color{blue} Midagi veel? Peab uurima.}
\end{enumerate}

\section*{\"Ulesande matemaatiline p\"ustitus}


Meil on muutujad $K_1,K_2,...,K_n, K_i \in \{1,2,3,4,5\}$, mis omavad tähendust kui vastused uuritava k\"usitluse k\"usimustele. Toome sisse ka tähistused $p_{i \alpha}, i \in \{1,2,...,n\}, \alpha \in \{1,2,3,4,5\}$, kus $p_{i \alpha}$ tähistab tõenäosust, et k\"usimusele $K_i$ anti vastus $\alpha$. 

Kujutama väärtused $K_1,K_2,...,K_n$ muutujateks  $L_1, L_2,L_2,...,L_n$, tuues sisse väärtused $\Gamma_i = \{\gamma_{1i},\gamma_{2i},\gamma_{3i},\gamma_{4i},\gamma_{5i}\} ~ i \in {1,2,...n}$, nii et kehtib järgnev : 

\begin{equation*}
K_i = 1 \implies L_i = \gamma_{i1}, K_i = 2 \implies L_i = \gamma_{i2},\cdots, K_i = 5 \implies L_i =\gamma_{i5}
\end{equation*}


Lisame kitsendused:
\begin{equation}
E(L_i) = p_{i1}*\gamma_{i1}+p_{i2}*\gamma_{i2}+p_{i3}*\gamma_{i3}+
p_{i4}*\gamma_{i4}+p_{i5}*\gamma_{i5}=0
\end{equation}

\begin{equation}
D(L_i) = 
p_{i1}*(\gamma_{i1})^2+ p_{i2}*(\gamma_{i2})^2 + p_{i3}*(\gamma_{i3})^2 + p_{i4}*(\gamma_{i4})^2 + p_{i5}*(\gamma_{i5}) = 1 
\end{equation}

Olgu meil tõenäosuste maatriks $P$:
\begin{tightcenter}
\begin{equation*}
P =
\begin{pmatrix}
p_{(11)(11)}&p_{(11)(12)}&p_{(11)(13)}&p_{(11)(14)}&p_{(11)(15)}&p_{(11)(21)}&\cdots&p_{(11)(n5)} \\
p_{(12)(11)}&p_{(12)(12)}&p_{(12)(13)}&p_{(12)(14)}&p_{(12)(15)}&p_{(12)(21)}&\cdots&p_{(12)(n5)} \\
\vdots&\vdots&\vdots&\vdots&\vdots&\vdots&\ddots&\vdots \\
p_{(n5)(11)}&p_{(n5)(12)}&p_{(n5)(13)}&p_{(n5)(14)}&p_{(n5)(15)}&p_{(n5)(21)}&\cdots&p_{(n5)(n5)} \\
\end{pmatrix} 
\end{equation*}
\end{tightcenter}

kus  $p_{(i \alpha) (j \beta)}, i,j \in \{1,2,...,n\}, \alpha , \beta \in \{1,2,3,4,5\}$ tähistab tõenäousust, et k\"usimusele $K_i$ anti vastus $\alpha$ ja k\"usimusele $K_j$ anti vastus $\beta$. 

Paneme tähele, et kuna $p_{(i \alpha)( i \alpha)} = 1$, siis avaldub eelnev maatriks ka järgnevalt:

\begin{tightcenter}
\begin{equation*}
P =
\begin{pmatrix}
1&p_{(11)(12)}&p_{(11)(13)}&p_{(11)(14)}&p_{(11)(15)}&p_{(11)(21)}&\cdots&p_{(11)(n5)} \\
p_{(12)(11)}&1&p_{(12)(13)}&p_{(12)(14)}&p_{(12)(15)}&p_{(12)(21)}&\cdots&p_{(12)(n5)} \\
\vdots&\vdots&\vdots&\vdots&\vdots&\vdots&\ddots&\vdots \\
p_{(n5)(11)}&p_{(n5)(12)}&p_{(n5)(13)}&p_{(n5)(14)}&p_{(n5)(15)}&p_{(n5)(21)}&\cdots&1\\
\end{pmatrix} 
\end{equation*}
\end{tightcenter}


Defineerime  vektori $x$:

\begin{tightcenter}
\begin{equation*}
x = (\gamma_{11},\gamma_{12},\gamma_{13} ,\gamma_{14},\gamma_{15},\gamma_{21},\gamma_{22},\gamma_{23},\gamma_{24},\gamma_{25}, \cdots ,\gamma_{n1},\gamma_{n2},\gamma_{n3},\gamma_{n4},\gamma_{n5})
\end{equation*}
\end{tightcenter}


Siis $xPx^T = \sum \limits_{i=1}^n \sum \limits_{j=1}^n COV(L_i,L_j)$. Veendume selles:
\begin{tightcenter}
\begin{equation*}
\begin{gathered}
xPx^T =
\begin{pmatrix}
\gamma_{11} & \gamma_{12} & \cdots & \gamma_{n5} 
\end{pmatrix}
\begin{pmatrix}
p_{(11)(11)}&p_{(11)(12)}&\cdots&p_{(11)(n5)} \\
p_{(12)(11)}&p_{(12)(12)}&\cdots&p_{(12)(n5)} \\
\vdots&\vdots&\ddots&\vdots \\
p_{(n5)(11)}&p_{(n5)(12)}&\cdots&p_{(n5)(n5)} \\
\end{pmatrix} 
\begin{pmatrix}
\gamma_{11} \\
\gamma_{12} \\
\vdots \\
\gamma_{n5}
\end{pmatrix}
= \\
= 
\begin{pmatrix}
\sum \limits_{j=1}^n \sum \limits_{l=1}^5 \gamma_{jl}p_{(jl)(11)}& \sum \limits_{j=1}^n \sum \limits_{l=1}^5 \gamma_{jl}p_{(jl)(12)} & \cdots &  \sum \limits_{j=1}^n \sum \limits_{l=1}^5 \gamma_{jl}p_{(jl)(n5)} \\
\end{pmatrix}
\begin{pmatrix}
\gamma_{11} \\
\gamma_{12} \\
\vdots \\
\gamma_{n5}
\end{pmatrix}
=\\
=
\sum \limits_{j=1}^n \sum \limits_{l=1}^5 \gamma_{jl}p_{(jl)(11)} + \sum \limits_{j=1}^n \sum \limits_{l=1}^5 \gamma_{jl}p_{(jl)(12)} + \cdots +  \sum \limits_{j=1}^n \sum \limits_{l=1}^5 \gamma_{jl}p_{(jl)(n5)}=\\
= \sum \limits_{i=1}^{n} \sum \limits_{k=1}^{5} \sum \limits_{j=1}^n \sum \limits_{l=1}^5 \gamma_{jl}p_{(jl)(ik)} \gamma_{ik} 
= \sum \limits_{i=1}^{n}  \sum \limits_{j=1}^n \sum \limits_{k=1}^{5} \sum \limits_{l=1}^5 \gamma_{jl}p_{(jl)(ik)} \gamma_{ik} =
\sum \limits_{i=1}^{n} \sum \limits_{j=1}^n E(L_iL_j) = \\
\underset{(1)}{=} \sum \limits_{i=1}^{n} \sum \limits_{j=1}^n E(L_iL_j) - E(L_i)E(L_j) = \sum \limits_{i=1}^{n} \sum \limits_{j=1}^n COV(L_i,L_j)
\end{gathered}
\end{equation*}
\end{tightcenter}



Selle põhjal saame moodustada \textit{Quadratically constrained quadratic programm}-i, kus \textit{objective function} {\color{cyan} (Eesti keels kulufunktsioon?)} on $max ~ x^TPx$ ja piiravateks fukntsioonideks on funktsioonid $R_i^Tx = 0,  i \in {1,2,...,n}$, kus $R_i = (\underbrace{0,0,...0,0}_{(i-1)*5}p_ia,p_ib,p_ic,p_id,p_ie,\underbrace{0,0,...,0,0}_{(n-i)*5})$ ja funktsioonid $x^TP_ix = 1, i \in {1,2,...,n}$, kus 

\begin{equation*}
P_i =
\begin{pmatrix}
p_{ja}&0&\cdots &0 \\
0&p_{jb}&\cdots &0 \\
\vdots & \vdots & \ddots & \vdots \\
0&0&\cdots & p_{je} \\
\end{pmatrix}
,~p_j\alpha =
\begin{cases} 
0 &  j \neq i \\ 
p_i\alpha & i = j 
\end{cases}
, \alpha \in \{a,b,c,d,e\}
\end{equation*}


\pagebreak
\section{Lisa}

\begin{figure}[H]
\centering
\includegraphics[width=0.8\textwidth]{ois_tagasiside_toodeldud.png}
\caption{Näide Tartu \"Ulikooli õppeinfo s\"usteemi tagasiside ankeedist, kus rakendatakse Likerti skaalat \cite{UT}}
\label{likert1}
\end{figure}

\begin{figure}[H]
\centering
\includegraphics[width=0.5\textwidth, height = 0.7\textwidth]{Example_Likert_Scale.png}
\caption{Näide k\"usimustikust, kus rakendatud Likerti skaalat, k\"usimused on paigutatud nende järjestikulisus rõhutamiseks teljele\cite{Smith}}
\label{likert2}
\end{figure}

\pagebreak
\bibliography{mata_baka}{}
\bibliographystyle{plain}


\end{document}