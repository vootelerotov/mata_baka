\documentclass{article}

\usepackage[utf8]{inputenc}
\usepackage{enumerate}

\author{Vootele Rõtov}
\title{Bakatöö}
\begin{document}

\section*{Miks ma käesolevat asja teen?}

Panen kirja mõned põhjused, miks ma tegelen selle asjaga:
\begin{enumerate}[I]
\item Sellest on kellegile kasu - loodan realselt, et saan hakkama mingi toreda asjaga, millest keegi(näiteks Marguse psühholoogidest sõbrad) kasu saab.
\item Saab kätte selle paberi, mis teeb minust parema inimese. 
\item Midagi uut, olen juba päris pikalt tarkust ühes formaadis kuula õppejõudu, tööta läbi tema poolt valitud materjalid, esita see õppejõule, äkki selline formaat, kus tuleb ise otsida ja ise mõelda meeldib.
\item Väljakutse kah, kuigi viimaste aastatega on enesemotivatsioon kõvasti paranenud, ei ole see veel seal, kus ta olla võiks. Loodetavasti "treenin" seda aspekti.
\end{enumerate}

\section*{Mida me teen?}

Meil on valikvastutsega küsimustik, kõik küsimusid 5 palli skaalal. Vaja oleks mudelit, mille põhjal hinnata ühe küsimuse vastust ülejäänud küsimuste põhjal.

Põhimõteliselt on mitmeid erinevaid lahendusi. Margus pakkus välja järgmise mõtte : 

\textit{Kolmas mõte oleks aga lineariseerida muutuja valides kordajad "optimaalselt" nii, et kokku oleks (lineaarsete s.o. pearsoni) korellatsioonide summa maksimaalne.}

Minu \"ulesanne: analüüsida, kas see lahendus idee on mõistlik ja kui on mõislik, siis välja mõelda, kuidas see lahendus täpselt võiks välja näha. 



\end{document}